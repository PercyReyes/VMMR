\documentclass[a4paper, 16pt]{article}

% packages
\usepackage{fancyhdr}
\usepackage{setspace}
\usepackage[style=authoryear-ibid,backend=biber]{biblatex}

\begin{document}

\begin{titlepage}
 \title{COP507 - Computer Vision \& Embedded Systems}
 \date{29th December 2017}
 \author{ B717426}
 \maketitle 
\end{titlepage}
 
 \newpage
 
 \tableofcontents{}
 
 \newpage
 %---------------- introduction ------------ %
 
 \section{Introduction}
 \onehalfspacing
 Make and Model Recognition (VMMR) has been a subject area of interest within computer vision both in industry and academia. The importance of the ability to recognise a variety of vehicles spans across different disciplines and applications such as driver assistance, traffic monitoring, law enforcement and surveillance. As stated by Dehghan et al (2017), this categorisation task has been a problem for classical computing however, a result of great advances in artificial intelligence, this task has been made possible with accuracy rates above 90 percent. The task given was to design and implement a VMMR system with the capability to recognise the make and model of a vehicle based on only the front view (that is, headlights, grill and bumper). Based on the image of a vehicle fed as input, the output of the system should be the make and model of the vehicle. This report explains a practical use of the VMMR system that was implemented, furthermore, it explains the design of the system and any performance restrictions or performance enhancements that will be expected should the system implemented in hardware.
\break

There some basic processes that are essential for any computer vision application. These include pre-processing the acquired images, feature detection and feature extraction. Pre-processing involves making the image data conducive for feature extraction. Some of these processes includes noise reduction, morphological operations, colour correction and image re-sampling. Feature extraction is the process by which key points are obtained from the image. These key points consist of blobs, edges and corners.
 
 \newpage
 
 %--------------- Section 2: Preprocessing Tasks ------------------ % 
 \section{Preprocessing Tasks}
 This section discusses the pre-processing tasks that are required to implement the Vehicle Make and Model System in practical scenarios.
 
 \subsection{CCTV camera located at the side of a motor way}
 
 \subsubsection{CCTV camera placed at an angle}
Considering the camera is at the side of the motor way, one thing to consider is that the angle at which the camera is placed. Therefore, before performing any feature detection or extraction, the images taken from that camera must be set to the right orientation. This could be done by skewing the image or rotating the image as this will set the image to the right orientation. Traditional mathematical algorithms and formulas exist which can be used to correct the angle of rotation or the degree of skewing required. Considering that since an image can be represented as a matrix of pixel values, affine transformation matrices could be applied to the image to correct the problem (Gonzales \& Woods, 2008). 

\subsubsection{ Varying lighting conditions }
Another factor to consider is the lighting conditions. The reason being that, in the day time, pictures taken may or may not have the appropriate lighting. In the night time, the images taken may be very dark. Therefore, in order to make the images conducive for training, the images will need to be made brighter. A remedy to this problem could be applying a brightness adjustment algorithm. According to Szeliski (2010), this can be done by multiplying or adding the image by the constant. 

\begin{center}
 $g(x) = \alpha f(x) + \beta$
\end{center}

Where $\alpha$ and $\beta$ are parameters which influence the contrast and brightness of the image and where $\alpha$ is greater than zero. 

\subsubsection{ Motion blur}

Motion blur is a phenomenon which occurs when there are inconsistencies between the speed of the moving object and the rate at which the camera captures frames. The problem with motion blur is that, it attenuates the image signal. Removing the motion blur from the image involves two processes. Firstly, estimating the amount by which the image has been blurred and thereafter, recovering a more realistic image using  deconvolution (point-wise division in the frequency domain). Cho (2010) proposes a method of removing motion blur which makes use of iterative distributed reweighting as this method enhances the visual quality of reconstructed images. Cho suggests the estimation of the blur kernel by examining edges within the blurred image. The edges within the blurred image encode estimations of the blur kernel from which the blur can be removed using the inverse radon transform. Furthermore, the method proposed by Cho (2010).

\end{document}
